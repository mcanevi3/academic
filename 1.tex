\section*{1. Proof of the LQR Optimal Controller}
\label{sec:lqr-proof}

\subsection*{The Regulation Problem}
Let a linear time-invariant (LTI) system be given in state-space form as:
\begin{align*}
\dot{x} &= Ax + Bu, \\
y &= Cx + Du,
\end{align*}
where $x \in \mathbb{R}^n$, $u \in \mathbb{R}^m$, $y \in \mathbb{R}^p$, with $A \in \mathbb{R}^{n \times n}$, $B \in \mathbb{R}^{n \times m}$, $C \in \mathbb{R}^{p \times n}$, and $D \in \mathbb{R}^{p \times m}$.

The Linear Quadratic Regulator (LQR) problem seeks a control input $u(t)$ that minimizes the infinite-horizon cost functional:
\begin{align*}
J = \frac{1}{2} \int_0^\infty \left( x(t)^T Q x(t) + u(t)^T R u(t) \right) dt,
\end{align*}
where $Q \in \mathbb{R}^{n \times n}$ is symmetric positive semi-definite ($Q \succeq 0$), and $R \in \mathbb{R}^{m \times m}$ is symmetric positive definite ($R \succ 0$).

\hrulefill

%%% Step 1: Define the Augmented Lagrangian
We define the Lagrangian \( L(x, u, \dot{x}, \lambda) \) incorporating the dynamics as a constraint via the co-state vector $\lambda(t)$:
\begin{align*}
L(x,u,\dot{x},\lambda) = \frac{1}{2} x^T Q x + \frac{1}{2} u^T R u + \lambda^T (Ax + Bu - \dot{x}).
\end{align*}

\hrulefill

%%% Step 2: Apply the Euler-Lagrange Equations
The necessary conditions for optimality from calculus of variations are:
\begin{align*}
\frac{d}{dt} \left( \frac{\partial L}{\partial \dot{x}} \right) - \frac{\partial L}{\partial x} &= 0, \\
\frac{\partial L}{\partial u} &= 0, \\
\frac{\partial L}{\partial \lambda} &= 0.
\end{align*}

Compute the partial derivatives:
\begin{align*}
\frac{\partial L}{\partial \dot{x}} &= -\lambda, \\
\frac{\partial L}{\partial x} &= Qx + A^T \lambda, \\
\frac{\partial L}{\partial u} &= R u + B^T \lambda, \\
\frac{\partial L}{\partial \lambda} &= Ax + Bu - \dot{x}.
\end{align*}

Substitute into the optimality conditions:
\begin{align*}
\dot{\lambda} &= -Qx - A^T \lambda, \quad \text{(costate dynamics)} \\
0 &= R u + B^T \lambda, \quad \text{(stationarity w.r.t. } u) \\
0 &= Ax + Bu - \dot{x}. \quad \text{(primal dynamics)}
\end{align*}

\hrulefill

%%% Step 3: Optimal Control Law
From stationarity:
\begin{align*}
u = -R^{-1} B^T \lambda.
\end{align*}

Assume the co-state vector is a linear function of the state:
\begin{align*}
\lambda = P x,
\end{align*}
where \( P \in \mathbb{R}^{n \times n} \) is a symmetric matrix to be determined.

Differentiate both sides:
\begin{align*}
\dot{\lambda} = P \dot{x} = P(Ax + Bu).
\end{align*}

Substitute the control law:
\begin{align*}
\dot{\lambda} = P (A - B R^{-1} B^T P) x.
\end{align*}

From the costate dynamics:
\begin{align*}
\dot{\lambda} = -Q x - A^T \lambda = -Q x - A^T P x.
\end{align*}

Equating both expressions for $\dot{\lambda}$:
\begin{align*}
P (A - B R^{-1} B^T P) x = -Q x - A^T P x.
\end{align*}

Since this holds for all $x$, we equate coefficients:
\begin{align*}
P A - P B R^{-1} B^T P = -Q - A^T P,
\end{align*}
which simplifies to the Algebraic Riccati Equation:
\begin{align*}
A^T P + P A - P B R^{-1} B^T P + Q = 0.
\end{align*}

\hrulefill

%%% Step 4: Final Result
The optimal state-feedback control law is:
\begin{align*}
u(t) = -K x(t), \quad \text{where} \quad K = R^{-1} B^T P,
\end{align*}
and \( P \) is the unique positive semi-definite solution to the Algebraic Riccati Equation:
\begin{align*}
A^T P + P A - P B R^{-1} B^T P + Q = 0.
\end{align*}
\hrulefill

%%% Step 5: Reference Tracking
\subsection*{Reference Tracking Problem}
For reference tracking the error and its integral are defined:
\begin{align*}
e(t) = r(t) - y(t),
\end{align*}
and
\begin{align*}
\dot{q}(t)= e(t)
\end{align*}
respectively. The cost function is modified as follows:
\begin{align*}
J=\frac{1}{2}\int_{0}^{\infty} \left(\begin{bmatrix}x\\q
\end{bmatrix}^T\begin{bmatrix}Q & 0\\0 & Q_q
\end{bmatrix}\begin{bmatrix}x\\q
\end{bmatrix} + u^T R u \right) dt
\end{align*}
where \( Q_q \) is a positive semi-definite matrix that penalizes the tracking error. The augmented system is given by:
\begin{align*}
\begin{bmatrix}\dot{x}\\\dot{q}
\end{bmatrix}=\begin{bmatrix}A & 0\\-C & 0
\end{bmatrix}\begin{bmatrix}x\\q
\end{bmatrix}+\begin{bmatrix}B\\-D
\end{bmatrix}u+\begin{bmatrix}0\\I
\end{bmatrix}r\\
y=\begin{bmatrix}C & 0\end{bmatrix}\begin{bmatrix}x\\q
\end{bmatrix}+Du
\end{align*}