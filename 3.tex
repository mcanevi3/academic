The following can be stated for the closed-loop system
\begin{equation}
    (\lambda_1 I-(A-BK))R_1=0
\end{equation}
where $R_1$ is the eigenvector for eigenvalue $\lambda_1$. Let $K$ be given as
\begin{equation}
    K=-MR^{-1}
\end{equation}
then
\begin{equation}
\begin{split}
    (\lambda_1 I-(A-BM_1R_1^{-1}))R_1=0\\
    \lambda_1 R_1-AR_1+BM_1=0\\
    (\lambda_1I-A)R_1+BM_1=0\\
\end{split}
\end{equation}
Let $R_1\triangleq X_1n_1$ and $M_1\triangleq Y_1n_1$,
\begin{equation}
    \begin{split}
        (\lambda_1I-A)X_1n_1+BY_1n_1=0\\
        \begin{bmatrix}
            \lambda_1I-A& B
        \end{bmatrix}\begin{bmatrix}
            X_1n_1\\ Y_1n_1
        \end{bmatrix}=0
    \end{split}
\end{equation}
where $n_1$ is the linear combination vector. Choosing
\begin{equation}
    X_1n_1=R_{des}(1)
\end{equation}
hence,
\begin{equation}
    n_1=X_1^{-1}R_{des}(1)
\end{equation}
and $\mathcal{N}$ is the null space in
\begin{equation}
    \begin{bmatrix}
        \lambda_1I-A& B
    \end{bmatrix}\mathcal{N}=0
\end{equation}
The algorithm is as follows:
\begin{enumerate}
    \item Calculate the null space($\mathcal{N}$) of 
    \begin{equation}
        \begin{bmatrix}
            \lambda_1I-A& B
        \end{bmatrix}
    \end{equation}
    \item Partition the null space matrix $\mathcal{N}$ as 
    \begin{equation}
        \mathcal{N}=\begin{bmatrix}X_1n_1\\Y_1n_1\end{bmatrix}
    \end{equation}
    \item Calculate $n_1=X_1^{-1}R_{des}(1)$ and set $M_1=Y_1n_1$ and $R_1=X_1n_1$.
    \item Repeat steps for each eigenvalue.
    \item Determine controller gain $K=-MR^{-1}$.
\end{enumerate}

Since $R$ can be used to diagonalize the closed-loop matrix
\begin{equation}
    A+BK=R\Lambda R^{-1}
\end{equation}
this fact can be used to evaluate the following
\begin{equation}
    e^{(A+BK)t}=Re^{\Lambda t} R^{-1}
\end{equation}
where $R$ and $\Lambda$ are closed-loop parameters and are known. Thereby,
\begin{equation}
    x(t)=Re^{\Lambda t} R^{-1}x_0
\end{equation}
Defining $z=Rx$ and using the relation $x=R^{-1}z$ gives,
\begin{equation}
\begin{split}
    x(t)&=Re^{\Lambda t} R^{-1}x_0\\
    R^{-1}x(t)&=e^{\Lambda t} R^{-1}x_0\\
    z(t)&=e^{\Lambda t} z_0
\end{split}
\end{equation}
and the control law is updated as 
\begin{equation}
\begin{split}
    u&=Kx\\
    u&=KR^{-1}z
\end{split}
\end{equation}

Sky-hook controller is defined as 
\begin{equation}
    u=k\begin{bmatrix}0& 1& 0& 0\end{bmatrix}x
\end{equation}

The LQR controller is designed via solving the Algebraic Riccati Equation(ARE) 
\begin{equation}
    A^TP+PA+PBR^{-1}B^TP+Q=0
\end{equation}
or by using the Hamiltonian given as
\begin{equation}
    \mathcal{H}=\begin{bmatrix}
        A& BR^{-1}B\\
        -Q& -A^T
    \end{bmatrix}
\end{equation}
in the equation
\begin{equation}
    \mathcal{H}\begin{bmatrix}
        U_1\\U_2
    \end{bmatrix}=0
\end{equation}
if $\mathcal{H}$ has no eigenvalues on the $jw$-axis. The solution $P$ of the ARE is obtained via,
\begin{equation}
    P=U_2 U_1^{-1}
\end{equation}
and the lqr controller then is obtained using $K=-R^{-1}B^TP$. The Schur Complement 
\begin{equation}
    A-BD^{-1}C,\,
\begin{bmatrix}
    A& B\\C& D
\end{bmatrix}
\end{equation}
is used as follows,
\begin{equation}
\begin{split}
    A^TP+PA+PBR^{-1}B^TP+Q=0\\
    \begin{bmatrix}
        A^TP+PA+Q& -B\\
        B^T& R
    \end{bmatrix}
\end{split}
\end{equation}



