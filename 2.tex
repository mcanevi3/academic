A dynamical system is given as
\begin{equation}
    \begin{cases}
    \dot{x}=
    \begin{bmatrix}
        0& 1\\0& 0
    \end{bmatrix}
    x+
    \begin{bmatrix}
        0\\1
    \end{bmatrix}u&\\
    x(0)=x_0&
    \end{cases}
\end{equation}
where the control signal is constrained with 
\begin{equation}
    -1\leq u\leq 1
\end{equation}
Let the LQR weights be given as follows
\begin{equation}
    Q=\begin{bmatrix}
        1& 0\\0& 1
    \end{bmatrix}\quad R=r,\,r>0
\end{equation}
and the unknown $P=P^T$ is defined as
\begin{equation}
    P=\begin{bmatrix}
        p_1& p_2\\p_2& p_3
    \end{bmatrix}
\end{equation}
then the Algebraic Riccati Equation(ARE) is stated as follows,
\begin{equation}
\begin{split}
    A^TP+PA-PBR^{-1}B^TP+Q=0\\
    \begin{bmatrix}
        0& 0\\1& 0
    \end{bmatrix}\begin{bmatrix}
        p_1& p_2\\p_2& p_3
    \end{bmatrix}+\begin{bmatrix}
        p_1& p_2\\p_2& p_3
    \end{bmatrix}\begin{bmatrix}
        0& 1\\0& 0
    \end{bmatrix}-\begin{bmatrix}
        p_1& p_2\\p_2& p_3
    \end{bmatrix}\begin{bmatrix}
        0\\1
    \end{bmatrix}\frac{1}{r}\begin{bmatrix}
        0& 1
    \end{bmatrix}\begin{bmatrix}
        p_1& p_2\\p_2& p_3
    \end{bmatrix}+\begin{bmatrix}
        1& 0\\0& 1
    \end{bmatrix}=0\\
    \begin{bmatrix}
        r-p_2^2& -p_2p_3+p_1r\\-p_2p_3+p_1r& r-p_3^2+2p_2r
    \end{bmatrix}=0\\
    \begin{bmatrix}
        p_1\\
        p_2\\
        p_3
    \end{bmatrix}=\begin{bmatrix}
        \pm\sqrt{(1\pm2\sqrt{r})}\\
        \pm\sqrt{r}\\
        \pm\sqrt{r(1\pm2\sqrt{r})}
    \end{bmatrix}
\end{split}
\end{equation}
Since $P$ needs to be positive definite, $p_1>0$ and
\begin{equation}
\begin{split}
\begin{bmatrix}
    p_1& p_2\\p_2& p_3
\end{bmatrix}\succ 0\\
p_1p_3-p_2^2>0\\
(\pm\sqrt{(1\pm2\sqrt{r})})(\pm\sqrt{r(1\pm2\sqrt{r})})-r>0\\
\pm(1\pm2\sqrt{r})-\sqrt{r}>0\\
\pm1+\sqrt{r}>0,\quad\pm1-3\sqrt{r}>0
\end{split}
\end{equation}
needs to be satisfied, hence feasible choices for $r$ are 
\begin{equation}
    (1+\sqrt{r}),(1-3\sqrt{r}),(-1+\sqrt{r})
\end{equation}
$P$ is obtained as
\begin{equation}
P=\begin{bmatrix}
    \sqrt{(1\pm2\sqrt{r})}& \pm\sqrt{r}\\
    \pm\sqrt{r}& \sqrt{r(1\pm2\sqrt{r})}
\end{bmatrix}
\end{equation}
The controller gain $L=-RB^TP$ is obtained as 
\begin{equation}
\begin{split}
    L&=-RB^TP\\
    &=-r\begin{bmatrix}0& 1\end{bmatrix}\begin{bmatrix}p_1&p_2\\p_2& p_3\end{bmatrix}\\
    &=-r\begin{bmatrix}p_2& p_3\end{bmatrix}\\
    &=-r\begin{bmatrix}\pm\sqrt{r}& \sqrt{r(1\pm2\sqrt{r})}\end{bmatrix}\\
    &=\begin{bmatrix}\pm r\sqrt{r}& -r\sqrt{r(1\pm2\sqrt{r})}\end{bmatrix}\\
\end{split}
\end{equation}

The closed-loop system matrix is given 
\begin{equation}
\begin{split}
    \Phi&=e^{A+BL}\\
    &=e^{\begin{bmatrix}
    0& 1\\l_1& l_2
    \end{bmatrix}}
\end{split}
\end{equation}
The matrix exponent identity is given as
\begin{equation}
\begin{split}
    e^{tA}&=\frac{\alpha e^{\beta t}-\beta e^{\alpha t}}{\alpha-\beta}I+\frac{e^{\alpha t}- e^{\beta t}}{\alpha-\beta}A\\
    &=\begin{bmatrix}
        \frac{\alpha e^{\beta t}-\beta e^{\alpha t}}{\alpha-\beta}& 0\\
        0& \frac{\alpha e^{\beta t}-\beta e^{\alpha t}}{\alpha-\beta}
    \end{bmatrix}+\begin{bmatrix}
        0& \frac{e^{\alpha t}- e^{\beta t}}{\alpha-\beta}\\
        l_1\frac{e^{\alpha t}- e^{\beta t}}{\alpha-\beta}& l_2\frac{e^{\alpha t}- e^{\beta t}}{\alpha-\beta}
    \end{bmatrix}\\
    &=\begin{bmatrix}
        \frac{\alpha e^{\beta t}-\beta e^{\alpha t}}{\alpha-\beta}& \frac{e^{\alpha t}- e^{\beta t}}{\alpha-\beta}\\
        l_1\frac{e^{\alpha t}- e^{\beta t}}{\alpha-\beta}& l_2\frac{e^{\alpha t}- e^{\beta t}}{\alpha-\beta}+\frac{\alpha e^{\beta t}-\beta e^{\alpha t}}{\alpha-\beta}
    \end{bmatrix}\\
    &=\frac{1}{\alpha-\beta}\begin{bmatrix}
        \alpha e^{\beta t}-\beta e^{\alpha t}& e^{\alpha t}- e^{\beta t}\\
        l_1e^{\alpha t}-l_1e^{\beta t}& l_2e^{\alpha t}-l_2e^{\beta t}+\alpha e^{\beta t}-\beta e^{\alpha t}
    \end{bmatrix}
\end{split}
\end{equation}
The solution $x(t)$ is calculated as
\begin{equation}
\begin{split}
x_1(t)&=\frac{1}{\alpha-\beta}
(\alpha e^{\beta t}-\beta e^{\alpha t})x_1(0)
+\frac{1}{\alpha-\beta}(e^{\alpha t}- e^{\beta t})x_2(0)\\
x_2(t)&=\frac{1}{\alpha-\beta}
(l_1e^{\alpha t}-l_1e^{\beta t})x_1(0)
+\frac{1}{\alpha-\beta}(l_2e^{\alpha t}-l_2e^{\beta t}+\alpha e^{\beta t}-\beta e^{\alpha t})x_2(0)
\end{split}
\end{equation}
assuming $\alpha=\sigma+j\omega$ and $\beta=\sigma-j\omega$ and converted into 
\begin{equation}
\begin{split}
x_1(t)&=e^{\sigma t}\left(x_1(0)\cos{(\omega t)}+\frac{x_2(0)-\sigma x_1(0)}{\omega}\sin{(\omega t)}\right)\\
x_2(t)&=e^{\sigma t}\left(x_2(0)\cos{(\omega t)}+\frac{l_1x_1(0)+(l_2-\sigma) x_2(0)}{\omega}\sin{(\omega t)}\right)
\end{split}
\end{equation}
The control signal is obtained as follows
\begin{equation}
    \begin{split}
    u(t)&=\begin{bmatrix}l_1&l_2\end{bmatrix}\begin{bmatrix}x_1(t)\\x_2(t)\end{bmatrix}\\
    &=e^{\sigma t}\cos{(\omega t)}\left(
        x_1(0)l_1+x_2(0)l_2
    \right)\\
    &+\frac{e^{\sigma t}}{\omega}\sin{(\omega t)}\left(
        x_1(0)(l_1l_2-\sigma l_1)+x_2(0)(l_1-\sigma l_2+l_2^2)
    \right)
    \end{split}
\end{equation}
The control signal limit is obtained as
\begin{equation}
    \begin{split}
    u(t)&=\frac{e^{\sigma t}}{\omega}\Big(
        x_1(0)\Big[
            l_1\omega\cos{(\omega t)}+(l_1l_2-\sigma l_1)\sin{(\omega t)}\Big]\\
        &+x_2(0)\Big[
        (l_1-\sigma l_2+l_2^2)\sin{(\omega t)}+l_2\omega\cos{(\omega t)}
        \Big]
    \Big)\\
    ||u(t)||&=\left|\left|\frac{1}{\omega}(x_1(0)\sqrt{l_1^2\omega^2+(l_1l_2-\sigma l_1)^2})
    +x_2(0)\sqrt{(l_1-\sigma l_2+l_2^2)^2+l_2^2\omega^2}\right|\right|\\
    &\leq \left|\left|\frac{1}{\omega}x_1(0)\sqrt{l_1^2\omega^2+(l_1l_2-\sigma l_1)^2}\right|\right|+\left|\left|\frac{1}{\omega}x_2(0)\sqrt{(l_1-\sigma l_2+l_2^2)^2+l_2^2\omega^2}\right|\right|\\
    &\leq \frac{|x_1(0)|}{\omega}\sqrt{l_1^2\omega^2+(l_1l_2-\sigma l_1)^2}+\frac{|x_2(0)|}{\omega}\sqrt{(l_1-\sigma l_2+l_2^2)^2+l_2^2\omega^2}\\
    \end{split}
\end{equation}

